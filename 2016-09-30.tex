% Преамбула
% Первая команда - подгрузка стиля
% команда - 
%   \имя_команды[необяз.параметр]{параметр1}...{параметрK}
\documentclass[12pt]{article}
% стандартные размеры - 10pt, 11pt, 12pt
%  1pt = (1/72)in  => 12pt ~ 4.2mm

% больший размеры
%  1) переключить размер шрифта в тексте
%  2) взять другой стиль:
%    article - extarticle
%    book    - extbook     и т.д.
%       extarticle - 14pt, 17pt, 20pt


% геометрия страницы - дополнительный пакет
%  парметры - в необязательном параметре 
%    через запятую
\usepackage[a4paper,margin=20mm]{geometry}
% Параметры:
%  1) размер страницы - a4paper, a5paper, a6paper...
%  2) поля: имя_поля=размер (число ед.измерения)
%     pt - пункты
%     in - дюймы
%     mm - миллиметры
%     cm - сантиметры
%       какое-то количество типографских единиц, которые 
%       нам не нужны
%     en - ширина буквы "n"
%     em - ширина буквы "m"
%     ex - высота буквы "x"
%      !!! привязаны к базовому шрифту !!!
%    имена полей:
%      lmargin, tmargin, rmargin, bmargin
%       (left, top, right, bottom)
%      lmargin=30mm, rmargin=10mm, tmargin=15mm, 
%      bmargin=15mm
%        vmargin - вертикальные поля, сверху и снизу
%        hmargin - горизонтальные полы, справа и слева
%        margin - все 4 поля
%  3) ориентация страницы:
%     по умолчанию - portrait - книжная
%      landscape - альбомная

% кодировка входного файла
\usepackage[utf8]{inputenc}

% особенности национального набора
\usepackage[russian]{babel}  %  Babel - Вавилон


% Основной раздел - текст
\begin{document}

% Абзацы отделяются пустой строкой или командой \par

\thispagestyle{empty}
%\pagestyle{plain} - по умолчанию

    Светловолосый  мальчик  только  что  одолел  последний спуск со скалы и
теперь пробирался к лагуне. Школьный свитер он  снял  и  волочил  за  собой,
серая  рубашечка на нем взмокла, и волосы налипли на лоб. Шрамом врезавшаяся
в джунгли длинная  полоса  порушенного  леса  держала  жару,  как  баня.  Он
спотыкался  о  лианы  и  стволы, когда какая-то птица желто-красной вспышкой
взметнулась вверх, голося, как ведьма; и на ее крик эхом отозвался другой.

    - Эй, - был этот крик, - погоди-ка!

    Кусты возле просеки дрогнули, осыпая гремучий град капель.

    - Погоди-ка, - сказал голос. - Запутался я.

    Светловолосый мальчик  остановился  и  подтянул  гольфы  автоматическим
жестом, на секунду уподобившим джунгли окрестностям Лондона.

    Голос заговорил снова:

    - Двинуться не дают, ух и цопкие они!

    Тот,  кому  принадлежал  голос,  задом  выбирался  из  кустов, с трудом
выдирая у них свою грязную куртку. Пухлые голые ноги  коленками  застряли  в
шипах  и  были  все  расцарапаны.  Он  наклонился,  осторожно отцепил шипы и
повернулся. Он был ниже  светлого  и  очень  толстый.  Сделал  шаг,  нащупав
безопасную позицию, и глянул сквозь толстые очки.

    - А где же дядька, который с мегафоном?

    Светлый покачал головой:

    - Это  остров.  Так мне по крайней мере кажется. А там риф. Может, даже
тут вообще взрослых нет.

    Толстый оторопел:

    - Был же летчик. Правда, не в пассажирском отсеке  был,  а  впереди,  в
кабине.

    Светлый, сощурясь, озирал риф.

    - Ну,  а  ребята?  -  не унимался толстый. - Они же, некоторые-то, ведь
спаслись? Ведь же правда? Да ведь?

    Светлый мальчик пошел к  воде  как  можно  непринужденней.  Легко,  без
нажима  он  давал  понять  толстому,  что  разговор окончен. Но тот заспешил
следом.

    - И взрослых, их тут совсем нету, да?

    - Вероятно.

    Светлый произнес это мрачно. Но тотчас  его  одолел  восторг  сбывшейся
мечты.   Он  встал  на  голову  посреди  просеки  и  во  весь  рот  улыбался
опрокинутому толстому.

    - Без всяких взрослых!

    Толстый размышлял с минуту.

    - Летчик этот...

    Светлый сбросил ноги и сел на распаренную землю.

    - Наверно, нас высадил, а сам улетел.  Ему  тут  не  сесть.  Колеса  не
встанут.

    - Нас подбили!

    - Ну, он-то вернется еще, как миленький!

    Толстый покачал головой:

    - Мы  когда  спускались,  я  -  это - в окно смотрел, а там горело. Наш
самолет с другого края горел.

    Он блуждал взглядом по просеке.

    - Это все от фюзеляжа.

    Светлый  потянулся  рукой  и  пощупал  раскромсанный  край  ствола.  На
мгновенье он заинтересовался:

    - А что с ним стало? Куда он делся?

    - Волнами  сволокло. Ишь, опасно-то как, деревья все переломаты. А ведь
там небось ребята были еще.

    Он помолчал немного, потом решился.

    - Тебя как звать?

    - Ральф.

    Толстый ждал, что его в свою  очередь  спросят  об  имени,  но  ему  не
предложили  знакомиться;  светлый  мальчик,  назвавшийся  Ральфом, улыбнулся
рассеянно, встал и снова двинулся к лагуне. Толстый шел за ним по пятам.

    - Я вот думаю, тут еще много наших. Ты как - видал кого?

    Ральф покачал головой и ускорил шаг. Но наскочил на ветку и с  грохотом
шлепнулся.

    Толстый стоял рядом и дышал, как паровоз.

    - Мне  моя  тетя  не  велела бегать, - объяснил он, - потому что у меня
астма.

    - Ассы-ма-какассыма?

    - Ага. Запыхаюсь я. У меня у одного  со  всей  школы  астма,  -  сказал
толстый не без гордости. - А еще я очки с трех лет ношу.

    Он  снял  очки, протянул Ральфу, моргая и улыбаясь, а потом принялся их
протирать замызганной курткой. Вдруг его расплывчатые  черты  изменились  от
боли и сосредоточенности. Он утер пот со щек и поскорей нацепил очки на нос.

    - Фрукты эти...

    Он кинул взглядом по просеке.

    - Фрукты эти, - сказал он. - Вроде я...

    Он поправил очки, метнулся в сторонку и присел на корточки за спутанной
листвой.

    - Я сейчас...

    Ральф  осторожно  высвободился  и  нырнул  под  ветки. Сопенье толстого
тотчас осталось у него за  спиной,  и  он  поспешил  к  последнему  заслону,
отгораживавшему  его  от  берега.  Перелез  через  поваленный  ствол и разом
очутился уже не в джунглях.

    Берег был весь опушен пальмами. Они стояли, клонились, никли в лучах, а
зеленое оперенье висело в стофутовой выси. Под  ними  росла  жесткая  трава,
вспученная  вывороченными  корнями,  валялись гнилые кокосы и то тут, то там
пробивались новорожденные ростки. Сзади  была  тьма  леса  и  светлый  проем
просеки.  Ральф  замер,  забыв  руку  на  сером  стволе, и щурясь смотрел на
сверкающую воду. Там, наверное, в расстоянии мили лохматилась у  кораллового
рифа  белая  кипень  прибоя  и дальше темной синью стлалось открытое море. В
неровной дуге кораллов лагуна лежала тихо, как горное озеро  -  разнообразно
синее,  и  тенисто-зеленое,  и  лиловатое.  Полоска  песка  между  пальмовой
террасой и морем убегала тонкой лукой  неведомо  куда,  и  только  где-то  в
бесконечности  слева  от Ральфа пальмы, вода и берег сливались в одну точку;
и, почти видимая глазу, плавала вокруг жара.

    Он соскочил с террасы. Черные ботинки  зарылись  в  песок,  его  обдало
жаром.  Он  ощутил  тяжесть  одежды. Сбросил ботинки, двумя рывками сорвал с
себя гольфы. Снова вспрыгнул на террасу, стянул рубашку, стал среди больших,
как черепа, кокосов, в скользящих зеленых  тенях  от  леса  и  пальм.  Потом
расстегнул  змейку  на  ремне,  стащил  шорты и трусики и, голый, смотрел на
слепящую воду и берег.

    Он был достаточно большой, двенадцать с  лишним,  чтоб  пухлый  детский
животик  успел  подобраться;  но  пока  в  нем  еще  не ощущалась неловкость
подростка. По ширине и развороту плеч  видно  было,  что  он  мог  бы  стать
боксером,  если  бы  мягкость взгляда и рта не выдавала его безобидности. Он
легонько  похлопал  пальму  по  стволу  и,  вынужденный   наконец   признать
существование  острова,  снова  упоенно  захохотал  и  стал на голову. Ловко
перекувырнулся, спрыгнул на берег, упал на коленки, обеими руками подгреб  к
себе горкой песок. Потом выпрямился и сияющими глазами окинул воду.

    - Ральф...

    Толстый  мальчик осторожно спустил ноги с террасы и присел на край, как
на стульчик.

    - Я долго очень, ничего? От фруктов этих...

    Он протер очки и  утвердил  их  на  носу-пуговке.  Дужка  уже  пометила
переносицу четкой розовой галкой. Он окинул критическим оком золотистое тело
Ральфа,  потом  посмотрел  на  собственную  одежду. Взялся за язычок молнии,
пересекающей грудь.

    - Моя тетя...

    Но вдруг решительно дернул за молнию и потянул через голову всю куртку.

    - Ладно уж!

    Ральф смотрел на него искоса и молчал.

    - По-моему, нам надо все имена узнать, - сказал  толстый.  -  И  список
сделать. Надо созвать сбор.

    Ральф не клюнул на эту удочку, так что толстому пришлось продолжить.

    - А  меня  как  хочете  зовите - мне все равно, - открылся он Ральфу, -
лишь бы опять не обозвали, как в школе.

    Тут уж Ральф заинтересовался:

    - А как?

    Толстый огляделся, потом пригнулся к Ральфу. И зашептал:

    - Хрюша - во как они меня обозвали.

    Ральф зашелся от хохота. Даже вскочил.

    - Хрюша! Хрюша!

    - Ральф! Ну Ральф же!..

    Хрюша всплеснул руками в ужасном предчувствии:

    - Я сказал же, что не хочу...

    - Хрюша! Хрюша!

    Ральф выплясал на солнцепек, вернулся истребителем, распластав  крылья,
и обстрелял Хрюшу:

    - У-у-уф! Трах-тах-тах!

    Плюхнулся в песок у Хрюшиных ног и все заливался:

    - Хрюша!!

    Хрюша улыбался сдержанно, радуясь против воли хоть такому признанию.

    - Ладно уж. Ты только никому не рассказывай...

    Ральф хихикнул в песок.

    Снова на лице у Хрюши появилось выражение боли и сосредоточенности.

    - Минуточку...

    И он бросился в лес. Ральф поднялся и затрусил направо.

    Там   плавный   берег  резко  перебивала  новая  тема  в  пейзаже,  где
господствовала угловатость; большая площадка из  розового  гранита  напролом
врубалась  в  террасу и лес, образуя как бы подмостки высотой в четыре фута.
Сверху  площадку  припорошило  землей,  и  она  поросла  жесткой  травой   и
молоденькими   пальмами.   Пальмам  не  хватало  земли,  чтобы  как  следует
вытянуться, и,  достигнув  футов  двадцати  роста,  они  валились  и  сохли,
крест-накрест  перекрывая  площадку  стволами,  на которых очень удобно было
сидеть. Пока не рухнувшие пальмы распластали зеленую кровлю, с исподу всю  в
мечущемся  плетеве  отраженных  водяных  бликов.  Ральф подтянулся и влез на
площадку, в прохладу и сумрак, сощурил один глаз и решил, что тени у него на
плече в самом деле зеленые. Он прошел к краю площадки над морем и заглянул в
воду. Она была ясная до самого дна и вся расцвела тропическими водорослями и
кораллами.  Сверкающим  выводком  туда-сюда  носились  рыбешки.   У   Ральфа
вырвалось вслух на басовых струнах восторга:

    - Потряса-а-а!

    За  площадкой открылось еще новое чудо. Какие-то силы творенья - тайфун
ли то был или отбушевавшая уже у него на  глазах  буря  -  отгородили  часть
лагуны  песчаной косой, так что получилась глубокая длинная заводь, запертая
с дальнего конца отвесной стеной  розового  гранита.  Ральф,  уже  наученный
опытом,  не  решался  по  внешнему виду судить о глубине бухты и готовился к
разочарованью. Но остров не обманул, и немыслимая бухта,  которую,  конечно,
мог  накрыть  только  самый  высокий  прилив,  была  с  одного  бока до того
глубокая, что даже темно-зеленая. Ральф тщательно обследовал ярдов  тридцать
и  только  потом  нырнул.  Вода оказалась теплее тела, он плавал как будто в
огромной ванне.

    Хрюша снова был тут  как  тут,  сел  на  каменный  уступ  и  завистливо
разглядывал зеленое и белое тело Ральфа.

    - А ты ничего плаваешь!

    - Хрюша.

    Хрюша  снял  ботинки,  носки,  осторожно сложил на уступе и окунул ногу
одним пальцем.

    - Горячо!

    - А ты как думал?

    - Я вообще-то никак не думал. Моя тетя... \ldots
% - дефис - короткий и толстый, для многоосновных слов
%   частиц, переносы

кресло-качалка    как-то  кое-как как-нибудь

% -- короткое тире, en-тире
%  для диапазонов, для наборов фамилий

стр.27--108  теорему Штурма--Лиувилля  закон Менделеева--Клайперона

% --- длинное тире, em-тире,
%   знак препинания -  в неполных предложениях, в диалогах

Тире --- это знак препинания

,,кавычки-лапки'' --- две запятых, два апострофа (на букве э)

``американские кавычки'' --- два обратных апострофа (на букве ё), два прямых апострофа

''две верхних правых кавычки''

<<кавычки-ёлочки>> --- меньше-меньше, больше-больше

"кавычки ой" ой-ой-ой-ой!!!

    - Слыхали про твою тетю!

    Ральф нырнул и поплыл под водой  с  открытыми  глазами:  песчаный  край
бухты  маячил,  как  горный  кряж. Он зажал нос, перевернулся на спину, и по
самому лицу заплясали золотые осколки света. Хрюша с решительным видом  стал
стягивать  шорты.  Вот  он  уже  стоял  голый,  белый и толстый. На цыпочках
спустился по песку и сел по шею в воде, гордо улыбаясь Ральфу.  





birk\"{a}user

t\'{e}l\`{e}phone

gar\c{c}on Ba\c{s}car

\textrm{Оюычный текст}

\textit{Курсив}

\textbf{Жирный}

\textit{\textbf{Жирный курсив}}

\textbf{\textit{Жирный курсив}}

\textsc{Строчные - уменьшенный вид заглавных}

\texttt{Моноширинный шрифт --- ААА}



{\tiny Очень маленький}

{\footnotesize Маленький}

{\small не очень маленький}

{\normalsize Базавый}

{\large не очень большой }

{\Large  большой }

{\LARGE  очень большой }

{\huge  очень очень большой }

{\Huge Гигантский\footnotemark[4]}

1






%\addtocounter{имя счетчика}{сколько добавить}
%\setcounter{имя счетчика}{устанавливаемое значение}
аааааа\footnote{Это сноска}

%\usepackage{setspace} %для установки жестрочного интервала

{\tiny  %\setstretch{0.8}   
	Ральф нырнул и поплыл под водой  с  открытыми  глазами:  песчаный  край
бухты  маячил,  как  горный  кряж. Он зажал нос, перевернулся на спину, и по
самому лицу заплясали золотые осколки света. Хрюша с решительным видом  стал
стягивать  шорты.  Вот  он  уже  стоял  голый,  белый и толстый. На цыпочках
спустился по песку и сел по шею в воде, гордо улыбаясь Ральфу. 
Ральф нырнул и поплыл под водой  с  открытыми  глазами:  песчаный  край
бухты  маячил,  как  горный  кряж. Он зажал нос, перевернулся на спину, и по
самому лицу заплясали золотые осколки света. Хрюша с решительным видом  стал
стягивать  шорты.  Вот  он  уже  стоял  голый,  белый и толстый. На цыпочках
спустился по песку и сел по шею в воде, гордо улыбаясь Ральфу. 
Ральф нырнул и поплыл под водой  с  открытыми  глазами:  песчаный  край
бухты  маячил,  как  горный  кряж. Он зажал нос, перевернулся на спину, и по
самому лицу заплясали золотые осколки света. Хрюша с решительным видом  стал
стягивать  шорты.  Вот  он  уже  стоял  голый,  белый и толстый. На цыпочках
спустился по песку и сел по шею в воде, гордо улыбаясь Ральфу.  
\par}
\footnote[4]{Четвертая сноска}

%сецкионирование
\tableofcontents
%\chapter - глава , только для book , но не aticle
%\section - раздел
%\subsection 
%\subsubsection 
%\paragraph

\section{Глава о мальчике}
\subsection{Подглава}
\subsubsection{Подподглова}
\makeatletter
\renewcommand{\subsection}{\@startsection{subsection}{2}{0pt}{2\baselineskip}{\baselineskip}{\Large \itshape} }
\makeatother
\subsection{Подглава1}

\end{document}

